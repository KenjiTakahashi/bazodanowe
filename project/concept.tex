\documentclass[fontsize=14pt]{scrartcl}

\usepackage{polski}
\usepackage{a4wide}
\usepackage{graphicx}
\usepackage{epstopdf}
\usepackage{amsmath,amssymb}
\usepackage{bm}
\usepackage{float}
\usepackage{amsthm}
\usepackage{fontspec}
\usepackage{setspace}
\usepackage{titling}
\defaultfontfeatures{Scale=MatchLowercase}
\setmainfont[Ligatures=TeX]{Liberation Serif}
\setsansfont{Liberation Sans}
\setmonofont[SmallCapsFont={Liberation Mono}]{Liberation Mono}
\setlength{\droptitle}{-1.5in}

\title{\LARGE \textbf{{Projekt z Kursu programowania aplikacji bazodanowych}}}
\author{Karol Woźniak}
\date{Wrocław, dnia \today r.}

\begin{document}
\thispagestyle{empty}
\maketitle
\section{Co?}
Serwis wspomagający utrzymywanie porządku w zbiorach książkowych.\\
Innymi słowy: Kolekcjonowanie książek, uzupełnianie informacji, oznaczanie przeczytanych/pożyczonych, itp.
\section{Jak?}
Użytkownik posiadający konto w serwisie operuje na globalnej bazie książek, z której tworzy podzbiór zawierający jego książki. Może również, w razie potrzeby, dodawać do globalnej bazy nowe pozycje, bądź modyfikować istniejące.

Nad czystością bazy czuwają administratorzy, których zadaniem jest zatwierdzanie zmian dokonanych przez użytkowników oraz podejmowanie odpowiednich kroków wobec użytkowników ``niszczących'' bazę.\\
Mają oni również możliwość nadania użytkownikowi statusu ``zaufanego''. Modyfikacje dokonywane przez zaufanego użytkownika pojawiają się od razu po dodaniu, nie przechodząc dodatkowego poziomu moderacji.

Niezalogowani użytkownicy mogą przeglądać bazę w trybie tylko do odczytu.\\[1ex]

Wartością dodaną (jeśli zdąrzę) będzie automatyczne rozwiązywanie danych o książce na podstawie numeru ISBN (np. z bazy amazonu lub podobnej).
\section{Czym?}
Entity Framework + ASP.NET MVC, ``pure code first'', tj.\ baza w całości generowana z kodu źródłowego.
\end{document}
